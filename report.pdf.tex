\documentclass{article}%
\usepackage[T1]{fontenc}%
\usepackage[utf8]{inputenc}%
\usepackage{lmodern}%
\usepackage{textcomp}%
\usepackage{lastpage}%
\usepackage{geometry}%
\geometry{margin=1in}%
%
\usepackage{listings}%
\usepackage{color}%
\lstset{('language', 'Python')}{('basicstyle', '\textbackslash{}\textbackslash{}ttfamily\textbackslash{}\textbackslash{}small')}{('keywordstyle', '\textbackslash{}\textbackslash{}color\{blue\}\textbackslash{}\textbackslash{}bfseries')}{('stringstyle', '\textbackslash{}\textbackslash{}color\{red\}')}{('commentstyle', '\textbackslash{}\textbackslash{}color\{green!50!black\}')}{('numbers', 'left')}{('numberstyle', '\textbackslash{}\textbackslash{}tiny\textbackslash{}\textbackslash{}color\{gray\}')}{('stepnumber', '1')}{('numbersep', '5pt')}{('showspaces', 'false')}{('showstringspaces', 'false')}{('frame', 'single')}{('breaklines', 'true')}{('breakatwhitespace', 'true')}%
\title{Отчёт о проверке плагиата}%
\author{Система анализа кода}%
\date{\today}%
%
\begin{document}%
\normalsize%
\maketitle%
\section{Исходные файлы}%
\label{sec:}%
\subsection{Оригинальный файл}%
\label{subsec:}%
\lstinputlisting{\ttfamily{def sum_list(lst):
    total = 0
    for i in range(len(lst)):
        total += lst[i]
    return total

numbers = [1, 2, 3, 4, 5]
print(sum_list(numbers))
}}

%
\subsection{Проверяемый файл}%
\label{subsec:}%
\lstinputlisting{\ttfamily{def sum_list(lst):
    total = 0
    i = 0
    while i < len(lst):
        total += lst[i]
        i += 1
    return total

numbers = [1, 2, 3, 4, 5]
print(sum_list(numbers))
}}

%
\section{Результаты анализа}%
\label{sec:}%
Общий процент сходства кода: 62.19\%%
Уровень совпадения структуры: средний%
Сходство ключевых элементов: 69.57\%%
Совпадение последовательностей: 35.29\%%
Рекомендация: Код, скорее всего, является плагиатом.%
Пояснение: Общий процент сходства кода (62.19\%) высокий, что указывает на значительное совпадение. Ключевые элементы кода (переменные, функции) совпадают на 69.57\%, что может указывать на копирование. Последовательности кода совпадают на 35.29\%, что может быть признаком копирования. Уровень совпадения структуры кода средний, но это значение пока не влияет на результат из{-}за текущих ограничений анализа.

%
\end{document}